\documentclass{article}
\usepackage[utf8]{inputenc}
\usepackage{amsmath}
\usepackage{amsfonts}
\usepackage{bbding}
\title{Cuentas}
\author{fabian2121015 }
\date{September 2019}

\begin{document}

\maketitle

\section{Cuentas}
\begin{align}
m^{\prime } &= 4 \pi r^{2}\tilde{\rho}, \label{EinsEq300}  \\
\nu^{\prime} & = \frac{4\pi r^{3}\tilde{P}+ m}{r(r-2m)}, \label{EinsEq311} \\
8\pi \bar{P}_{\perp} &= \left\{ \left( 1-\frac{2m}{r} \right) \left( \nu^{\prime \prime} + (\nu^{\prime})^{2}+ \frac{\nu^{\prime}}{r}  \right) + \left(  \frac{m}{r^{2}} -\frac{m^{\prime }}{r}  \right) \left( \nu^{\prime} + \frac{1}{r}     \right)   \right\} 
\nonumber \\ 
& \qquad  + \frac{e^{-2\nu}}{4} \left\{ 2\left(1 - \frac{2m}{r} \right)^{-1}
\left( \frac{\ddot{m}}{r} +\frac{2\dot{m}^2}{r^2} \left(1 - \frac{2m}{r} \right)^{-1}
\right)\right. \nonumber \\
& \qquad + \left.
\frac{\dot{m}}{r}\left(1 - \frac{2m}{r} \right)^{-1}
\left(\frac{\dot{m}}{r}\left(1 - \frac{2m}{r} \right)^{-1} -\dot{\nu}
\right)
\right\} \; \textrm{and} \label{EinsEq322} \\
\dot{m} &= -\frac{4 \pi r^{2} e^{\nu-\lambda}}{1+\omega^{2}}\left( \omega(\tilde{\rho}+ \tilde{P})+ (1-\omega^{2})\mathcal{F}     \right)  \label{EinsEq301}\, ;
\end{align}
Los coeficientes de espín están definidos como
\begin{equation}
\gamma = -\dfrac{1}{2\sqrt{2}}\left( \dfrac{\dot{m}}{r}\left( 1-\dfrac{2m}{r}\right)^{-1} e^{-\nu} + \dfrac{4\pi r^{3}\bar{P}+m}{r \left( r-2m \right)}\left(1-\dfrac{2m}{r} \right)^{1/2}\right),
\label{Equation1}
\end{equation}
\begin{equation}
    \epsilon = \dfrac{1}{2\sqrt{2}}\left( \dfrac{\dot{m}}{r}\left( 1-\dfrac{2m}{r}\right)^{-1} e^{-\nu} - \dfrac{4\pi r^{3}\bar{P}+m}{r \left( r-2m \right)}\left(1-\dfrac{2m}{r} \right)^{1/2}\right)
\end{equation}
Evaluando (\ref{Equation1}) en la región anterior y posterior
\begin{align}
    \gamma_{+} = -\dfrac{1}{2\sqrt{2}}\left( \dfrac{\dot{m}_{+}}{r}\left( 1-\dfrac{2m_{+}}{r}\right)^{-1} e^{-\nu_{+}} + \dfrac{4\pi r^{3}\bar{P}_{+}+m_{+}}{r \left( r-2m_{+} \right)}\left(1-\dfrac{2m_{+}}{r} \right)^{1/2}\right),
    \label{Equation 3}
\end{align}
\begin{align}
    \gamma_{-} = -\dfrac{1}{2\sqrt{2}}\left( \dfrac{\dot{m}_{-}}{r}\left( 1-\dfrac{2m_{-}}{r}\right)^{-1} e^{-\nu_{-}} + \dfrac{4\pi r^{3}\bar{P}_{-}+m_{-}}{r \left( r-2m_{-} \right)}\left(1-\dfrac{2m_{-}}{r} \right)^{1/2}\right),
    \label{Equation 4}
\end{align}
Sumando las ecuaciones anteriores se obtiene
\begin{equation}
 \left[   \gamma \right]= -\dfrac{1}{2\sqrt{2}}\left[  \dfrac{\dot{m}}{r}\left( 1-\dfrac{2m}{r}\right)^{-1} e^{-\nu} \right]+ \left[\dfrac{4\pi r^{3}\bar{P}+m}{r \left( r-2m \right)}\left(1-\dfrac{2m}{r} \right)^{1/2}\right],
 \label{Equation 5}
\end{equation}
\begin{equation}
  \left[  \epsilon \right]= \dfrac{1}{2\sqrt{2}}\left[  \dfrac{\dot{m}}{r}\left( 1-\dfrac{2m}{r}\right)^{-1} e^{-\nu} \right]- \left[\dfrac{4\pi r^{3}\bar{P}+m}{r \left( r-2m \right)}\left(1-\dfrac{2m}{r} \right)^{1/2}\right],
  \label{Equation 5.1}
\end{equation}
Si se multiplica (\ref{Equation 3}) por $\dot{C}e^{-\nu_{+}}$ y (\ref{Equation 4}) por $\dot{C}e^{-\nu_{-}}$ se obtiene
\begin{equation}
    \gamma_{+} \dot{C} e^{-\nu_{+}} = -\dfrac{1}{2\sqrt{2}}\left( \dfrac{\dot{m}_{+}}{r}\left( 1-\dfrac{2m_{+}}{r}\right)^{-1} \dot{C}e^{-2\nu_{+}} + \dfrac{4\pi r^{3}\bar{P}_{+}+m_{+}}{r \left( r-2m_{+} \right)}\left(1-\dfrac{2m_{+}}{r} \right)^{1/2}\dot{C}e^{-\nu_{+}}\right), 
    \label{Equation 6}
\end{equation}
Y evaluando en la región anterior
\begin{equation}
    \gamma_{-} \dot{C} e^{-\nu_{-}} = -\dfrac{1}{2\sqrt{2}}\left( \dfrac{\dot{m}_{-}}{r}\left( 1-\dfrac{2m_{-}}{r}\right)^{-1} \dot{C}e^{-2\nu_{-}} + \dfrac{4\pi r^{3}\bar{P}_{-}+m_{-}}{r \left( r-2m_{-} \right)}\left(1-\dfrac{2m_{-}}{r} \right)^{1/2}\dot{C}e^{-\nu_{-}}\right), 
    \label{equation 7}
    \end{equation}
    Las ecuaciones (\ref{Equation 6}) y (\ref{equation 7}) dan el salto como 
    \begin{equation}
        \left[ \dot{C}\gamma e^{-\nu}\right]= -\dfrac{1}{2\sqrt{2}}\left[  \dfrac{\dot{m} \dot{C}}{r}\left( 1-\dfrac{2m}{r}\right)^{-1} e^{-2\nu} \right] + \left[\dfrac{4\pi r^{3}\bar{P}+m}{r \left( r-2m \right)}\left(1-\dfrac{2m}{r} \right)^{1/2}\dot{C}e^{-\nu}\right],
        \label{Equation 9.1}
    \end{equation}
     Lo mismo se hace con el coeficiente de espin $\epsilon$.
     \begin{equation}
        \left[ \dot{C}\epsilon e^{-\nu}\right]= \dfrac{1}{2\sqrt{2}}\left[  \dfrac{\dot{m} \dot{C}}{r}\left( 1-\dfrac{2m}{r}\right)^{-1} e^{-2\nu} \right] - \left[\dfrac{4\pi r^{3}\bar{P}+m}{r \left( r-2m \right)}\left(1-\dfrac{2m}{r} \right)^{1/2} \dot{C}e^{-\nu}\right],
        \label{Equation 9.2}
    \end{equation}
    Al sumarse (\ref{Equation 5}) y (\ref{Equation 5.1}) se obtiene
    \begin{equation}
        -\sqrt{2} \left(\left[ \gamma\right] + \left[ \epsilon \right] \right) = \left[\dfrac{4\pi r^{3}\bar{P}+m}{r \left( r-2m \right)}\left(1-\dfrac{2m}{r} \right)^{1/2}\right]
    \end{equation}
    y al restarse (\ref{Equation 9.1}) y (\ref{Equation 9.2}) da
    \begin{equation}
       - \sqrt{2}C \dot{C}\left( \left[ \gamma e^{-\nu}\right]- \left[ \epsilon e^{-\nu}\right]\right)= \left[  \dot{m}\dot{C}\left( 1-\dfrac{2m}{r}\right)^{-1} e^{-2\nu} \right]
    \end{equation}
    Si se suman las dos ecuaciones anteriores, se obtiene las condiciones de Rankine-hugoniot
    \begin{equation}
        \left[  \dot{m}\dot{C}\left( 1-\dfrac{2m}{r}\right)^{-1} e^{-2\nu} \right] + \left[\dfrac{4\pi r^{3}\bar{P}+m}{r \left( r-2m \right)}\left(1-\dfrac{2m}{r} \right)^{1/2}\right]= -\sqrt{2}C \dot{C}\left( \left[ \gamma e^{-\nu}\right]- \left[ \epsilon e^{-\nu}\right]\right)- \sqrt{2} \left(\left[ \gamma\right] + \left[ \epsilon \right] \right) 
    \end{equation}
    La otra condición de Rankine-Hugoniot se halla de la siguiente forma
    \begin{equation}
        \dot{m}(r,t)|_{C(t)}=\dot{m}(t,C)-\dot{C}m'(t,r)|_{C},
        \label{SecondRankine}
    \end{equation}
    Evaluando la función en una región antes y después
    \begin{equation}
        \left(\dot{m}(r,t)|_{C(t)}\right)_{+}=\left(\dot{m}(t,C)\right)_{+}-\dot{C}\left(m'(t,r)|_{C}\right)_{+},
        \label{Rankinesecond1}
    \end{equation}
    \begin{equation}
        \left(\dot{m}(r,t)|_{C(t)}\right)_{-}=\left(\dot{m}(t,C)\right)_{-}-\dot{C}\left(m'(t,r)|_{C}\right)_{-},
        \label{Rankinesecond2}
    \end{equation}
    Restando (\ref{Rankinesecond1}) y (\ref{Rankinesecond2}), utilizando la ecuación de Einstein 
    \begin{equation}
        m'= 4\pi r^{2}\bar{\rho},
    \end{equation}
    Con la ecuación de Einstein
    \begin{equation}
        \dot{m}_{+}+ 4\pi r^{2}\dot{C}\bar{\rho}_{+}= \left \{ m_{+}  \right\}_{0},
        \label{Masapoint1}
    \end{equation}
    \begin{equation}
        \dot{m}_{-}+ 4\pi r^{2}\dot{C}\bar{\rho}_{-}= \left \{ m_{-}  \right\}_{0},
        \label{MasaPoint2}
    \end{equation}
    
    se obtiene el salto en la segunda condición de Rankine Hugoniot
    \begin{equation}
        \left[ \dot{m}\right] + \left[ 4\pi r^{2}\dot{C}\bar{\rho} \right] - \left \{ \left[ m \right]  \right \}_{0}=0
    \end{equation}
    Si se sustituye las ecuaciones (\ref{Masapoint1}) y (\ref{MasaPoint2}) en las ecuaciones (\ref{Equation 6}) y (\ref{equation 7}) se obtiene
    \begin{equation}
        \gamma_{+} \dot{C} e^{-\nu_{+}} = -\dfrac{1}{2\sqrt{2}}\left( \dfrac{ \left \{ m_{+} \right\}_{0}-4\pi r^{2}\dot{C}\bar{\rho}_{+}}{r}\left( 1-\dfrac{2m_{+}}{r}\right)^{-1} \dot{C}e^{-2\nu_{+}} + \dfrac{4\pi r^{3}\bar{P}_{+}+m_{+}}{r \left( r-2m_{+} \right)}\left(1-\dfrac{2m_{+}}{r} \right)^{1/2}\dot{C}e^{-\nu_{+}}\right), 
    \end{equation}
     \begin{equation}
         \gamma_{-} \dot{C} e^{-\nu_{-}} = -\dfrac{1}{2\sqrt{2}}\left( \dfrac{ \left \{ m_{-}  \right\}_{0}-4\pi r^{2}\dot{C}\bar{\rho}_{-}}{r}\left( 1-\dfrac{2m_{-}}{r}\right)^{-1} \dot{C}e^{-2\nu_{-}} + \dfrac{4\pi r^{3}\bar{P}_{-}+m_{-}}{r \left( r-2m_{-} \right)}\left(1-\dfrac{2m_{-}}{r} \right)^{1/2}\dot{C}e^{-\nu_{-}}\right), 
     \end{equation}
    Al restarse las ecuaciones anteriores da
    \begin{align}
     \left[   \gamma \dot{C} e^{-\nu} \right]= -\dfrac{1}{2\sqrt{2}}\left[ \dfrac{ \left \{\left[ m \right]  \right\}_{0}-4\pi r^{2}\dot{C}\bar{\rho}}{r}\left( 1-\dfrac{2m}{r}\right)^{-1} \dot{C}e^{-2\nu} \right] + \\ \left[ \dfrac{4\pi r^{3}\bar{P}+m}{r \left( r-2m \right)}\left(1-\dfrac{2m}{r} \right)^{1/2}\dot{C}e^{-\nu}\right],
     \label{Equation111}
    \end{align}
    si se realiza las mismas operaciones con el coeficiente $\epsilon$ se obtiene
    \begin{align}
        \left[   \epsilon \dot{C} e^{-\nu} \right]= \dfrac{1}{2\sqrt{2}}\left[ \dfrac{ \left \{\left[ m \right]  \right\}_{0}-4\pi r^{2}\dot{C}\bar{\rho}}{r}\left( 1-\dfrac{2m}{r}\right)^{-1} \dot{C}e^{-2\nu} \right] - \\ \left[ \dfrac{4\pi r^{3}\bar{P}+m}{r \left( r-2m \right)}\left(1-\dfrac{2m}{r} \right)^{1/2}\dot{C}e^{-\nu}\right],
        \label{Equation211}
    \end{align}
    Al restarse las ecuaciones (\ref{Equation111}) y (\ref{Equation211}), se obtiene
    \begin{align}
        -\sqrt{2}C\dot{C}\left( \left[   \gamma  e^{-\nu} \right]-\left[   \epsilon  e^{-\nu} \right]\right)= \left[ \dfrac{ \left \{\left[ m \right]  \right\}_{0}-4\pi r^{2}\dot{C}\bar{\rho}}{r}\left( 1-\dfrac{2m}{r}\right)^{-1} \dot{C}e^{-2\nu} \right] 
    \end{align}
    \begin{align}
        -\sqrt{2}C\dot{C}\left( \left[   \gamma  e^{-\nu} \right]-\left[   \epsilon  e^{-\nu} \right]\right)= \left[ \left \{ m   \right\}_{0} \left( 1-\dfrac{2m}{r} \right)^{-1} \dot{C}e^{-2\nu}\right]- \left[4\pi r^{2}\dot{C}\bar{\rho}\left( 1-\dfrac{2m}{r}\right)^{-1} \dot{C}e^{-2\nu} \right] 
    \end{align}
    \begin{table}[htbp]
\begin{center}
\begin{tabular}{|l|l|l|}
\hline
superficie & $\left[m \right]=0$ & $\left[ \omega \right]=0$\\
\hline \hline \hline
Choque impulsivo & \XSolidBrush & \XSolidBrush\\ \hline
Capa & \XSolidBrush &\checkmark \\ \hline
Onda de choque & \checkmark & \XSolidBrush\\ \hline
Frontera & \checkmark & \checkmark \\ \hline
\end{tabular}
\caption{Continuidad de la masa y la velocidad del fluido en 
las diferentes superficies de acoplamiento }
\label{tabla:sencilla}
\end{center}
\end{table}
    
\end{document}